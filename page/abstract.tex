% 中文摘要頁
\begin{ZhAbstract}
    \begin{ZhAbstractItems}
        % 關鍵詞,請自己填,請自己填,多個關鍵字以逗號(、)隔開
        \noindent \text 關鍵詞:藍牙低功耗、物聯網、FruityMesh、BLE Mesh、網路拓撲、傳輸壅塞、封包重傳率、封包傳遞成功率

    \end{ZhAbstractItems}

    \begin{ZhAbstractDescription}
        藍牙低功耗(Bluetooth Low Energy, BLE)有省電及低成本的特性,使得藍牙技術在物聯網(Internet of Things, IoT)中占據重要的角色。在物聯網的應用中,大量使用無線感測網路(Wireless Sensor Networks, WSN),會在環境中分布建立許多的節點,而節點不只有當作感知器測量環境的數據,常常還要當作中繼節點,負責轉傳發送端與目的端之間的封包。最終,將所有量測的數據匯集到Sink節點,以監控所有的節點數據,將數據儲存後,進行分析後並做出適當的處理,也可以透過分析數據預測環境的變化,並提前做出適當的處理。
        
        本論文針對FruityMesh網路建立流程進行改善,讓Sink節點在網路建立完成後成為整個網狀網路的根節點,並且確保Sink節點斷線後重新連線時,仍然是根節點的角色。如此一來,所有非Sink節點在傳送封包至Sink節點時,只需要往父親節點傳輸封包即可,有效的減少節點發送及轉發的次數。
        
        此外,本論文亦針對網路中大量封包集中至根節點所導致的傳輸壅塞問題,提出一種基於拓樸層級進行分層式連線間隔(Connection Interval, CI)調整的機制。該機制依據節點距離根節點的跳數層級,賦予不同的連線間隔設定,愈接近根節點之中繼節點,CI的設定時間比較短,讓其傳輸頻率愈高,以提供較多的傳輸機會。此方法有效分攤了上層節點的傳輸負載,緩解因封包匯聚造成的壅塞,進而提升整體 Mesh 網路的封包傳遞成功率(Packet Delivery Ratio, PDR),並降低傳輸延遲與重傳率,改善網路效能。
    \end{ZhAbstractDescription}
    
\end{ZhAbstract}

