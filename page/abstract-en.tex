% 英文摘要頁
\begin{EnAbstract}
    \begin{EnAbstractItems}
        % 關鍵詞,請自己填,多個關鍵字以逗號 "," 隔開
        \noindent \text Keyword: Bluetooth Low Energy,Internet of Things,FruityMesh,BLE Mesh,Network Topology,Transmission Congestion,Retransmission Rate,Packet Delivery Ratio
    \end{EnAbstractItems}

    \begin{EnAbstractDescription}
        Bluetooth Low Energy (BLE) is characterized by its low power consumption and cost-effectiveness, making it a key technology in the development of the Internet of Things (IoT). In IoT applications, Wireless Sensor Networks (WSNs) are widely used, where numerous sensor nodes are distributed throughout the environment. These nodes not only collect environmental data but often function as relay nodes, responsible for forwarding packets between the source and destination. Ultimately, all collected data is aggregated at the Sink node, which serves as a central point for monitoring, storage, and analysis. This enables the system to make informed decisions and even predict environmental changes based on the analyzed data.
This thesis proposes an enhancement to the FruityMesh network formation process, ensuring that the Sink node becomes the root node of the mesh network once the network is established. Furthermore, it guarantees that the Sink node retains its root node role even after disconnection and reconnection. With this configuration, non-Sink nodes can transmit packets toward the Sink node by simply forwarding them to their parent node, effectively reducing the number of transmissions and retransmissions required. Additionally, by tuning specific BLE parameters, the proposed method alleviates network congestion caused by high packet traffic toward the root node. The experimental results show that the proposed improvements increase the Packet Delivery Ratio (PDR), reduce transmission delay and retransmission rates, and ultimately enhance overall network performance.

    \end{EnAbstractDescription}
    
\end{EnAbstract}

