\begin{ZhChapter}

\chapter{結論與未來工作}

\section{結論}
本研究旨在針對 FruityMesh 架構進行拓樸控制與傳輸機制的優化設計,特別著重於支援 DOT(Destination-Oriented Transmission)機制所需的拓樸先決條件,即需保證 Sink 節點始終作為整個 Mesh 網路的根節點。為此,本研究提出一套拓樸重建演算法並進行系統性模擬與實作驗證,以評估其可行性與效能。

首先,在拓樸控制部分,本研究透過模擬大規模藍牙 Mesh 網路的初始化流程,驗證所提出之拓樸重構機制可於各種隨機拓樸環境下成功將 Sink 節點設為 Mesh 網路的根節點。此外,亦模擬 Sink 節點於斷線後重新連入網路的情境,證明即使在動態變化環境下,該節點仍可重新成為根節點,維持 DOT 機制運作所需的結構一致性。雖然所提出機制在拓樸重建所需時間上略高於原生 FruityMesh,惟可提供更穩定與具控制性的網路結構,提升後續傳輸機制的可控性與預測性。

在傳輸效能方面,透過比較 HopsToSink 指標,實驗結果顯示本研究提出之拓樸機制在大多數情境下皆能有效縮短節點至 Sink 的跳數路徑,形成更平衡的 Mesh 結構,進而有助於資料傳輸效率的提升。

進一步於小規模實體環境中實作驗證,實驗亦證實在實際 BLE Mesh 網路中,無論在初始化或 Sink 斷線重連後,均可成功維持 Sink 為 Mesh 網路根節點之特性,印證該機制於實務應用中的可行性。

最後,本研究結合拓樸控制與資料傳輸優化,進一步於 DOT 機制下導入分層式 Connection Interval(CI)調整策略,根據節點在 Mesh 中之層級動態分配連線間隔。實驗結果顯示,與原生 FruityMesh 及參考文獻 \cite{112TIT00392032} 中的 DOST 機制相比,本方法可有效降低封包延遲與重傳率,並於不同封包負載條件下穩定維持高封包傳輸成功率(PDR),展現出良好的傳輸穩定性與網路效能。

綜合以上結果,本研究所提出之拓樸控制與 CI 調整機制可有效滿足 DOT 機制之結構需求,並在實體 BLE Mesh 網路中具備良好之穩定性與延展性。

\section{未來工作}
\subsection{大規模實體環境之驗證}
目前拓樸控制機制僅於小規模實體環境中進行實作驗證,雖然已成功達成將 Sink 節點設為整體 Mesh 根節點之目標,然而在更大規模節點數量與複雜拓樸環境下,拓樸穩定性及收斂效率仍需進一步實證。未來可透過擴充實體節點規模,評估所提出拓樸控制機制於大規模網路環境下之穩定性與延展性。

\subsection{動態參數調整機制設計}
在傳輸機制方面,現階段所採用之分層 CI(Connection Interval)配置為靜態設定,未能反映實際網路中動態變化的流量特性。實際應用中,不同節點可能因任務負載、感測頻率或應用情境不同而產生不均勻的封包傳輸需求。未來可進一步發展具自適應能力之機制,依據各層節點之流量狀況動態調整 CI 或 CE(Connection Event)參數,以提升網路對傳輸壅塞的反應能力與整體服務品質(Quality of Service, QoS)。此類設計將有助於實現更高效率、更穩定且具可擴展性的 BLE Mesh 通訊架構。


\end{ZhChapter}