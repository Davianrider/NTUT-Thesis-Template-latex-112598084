\begin{ZhChapter}

\chapter{BLE Mesh拓樸建立機制設計}

\section{問題分析}

\subsection{BLE Mesh拓樸建立問題}

在原生FruityMesh架構中,節點的建立與連線並未針對資料匯集或匯出端進行特別設計,因此整體網路並無明確的Root節點。其封包傳輸方式主要採用廣播機制,即每當節點產生封包時,會向所有相鄰節點廣播傳送,期望藉由鄰近節點的重傳將封包推進至目的地。雖然此方法具備一定的自我修復能力,但在多節點、大範圍的Mesh網路中,極易導致封包重複傳送與碰撞,進而產生所謂的「廣播風暴」現象,嚴重影響整體網路效能與穩定性。

為了解決此問題,\cite{112TIT00392032}研究中引入DOT機制,將原先的無向廣播傳輸改為目的導向的封包路由方式,透過樹狀拓樸的建立使資料傳輸路徑更具方向性與效率性。然而,導入DOT排程機制的前提之一,是整個BLE Mesh網路需具備清楚的階層結構,而Sink節點(資料匯集端)必須作為整體樹狀拓樸的根節點。唯有如此,封包才能沿著既定的父子節點路徑,自節點有效匯流至Sink節點,達成DOT排程所需的單向、無冗餘傳輸目標。

然而,由於FruityMesh原生設計並非以Sink節點為中心的拓樸為出發點,現有拓樸建立流程尚無法保證Sink節點必然成為根節點。在未進行拓樸控制調整的情況下,可能出現Sink位置處於樹葉節點甚至中繼節點等非理想情境。因此,若要在FruityMesh架構中有效實作DOT排程機制,勢必需重新設計拓樸建立邏輯,強制指定使Sink節點始終位於拓樸樹的根部,即使在斷線並重新連線的情況下,亦能恢復其根節點角色,確保網路穩定性與傳輸效能。

\subsection{BLE Mesh封包傳輸問題}

在基於FruityMesh的藍牙低功耗網狀網路傳輸機制設計中,\cite{112TIT00392032}研究已針對封包延遲與封包抵達率等品質指標進行優化,並取得初步成效。該研究藉由導入DOT與DOST排程機制,成功改善了大部分的傳輸延遲與資料完整性。然而,即使在優化機制下,網路中仍不時出現封包遺失(掉封包)與重傳現象,顯示目前的傳輸機制尚存在進一步提升的空間。

造成封包掉落與重傳的關鍵因素之一,在於BLE Mesh網路的流量集中特性。由於整體網路皆以Sink節點作為封包的最終目的地,所有資料封包最終皆會匯集至該節點,導致越接近Sink的節點需承擔更多的中繼與轉發任務。這種負載不均的現象將使中樞節點在短時間內接收到大量傳輸要求,導致中樞節點的Sent buffer資源迅速耗盡。一旦緩衝區爆滿,不僅無法即時處理新進封包,還可能造成排程延遲、封包掉落,進而引發重傳機制啟動,進一步加重網路負載。

為有效緩解此類壅塞現象,需從連線參數層級進行調整,特別是針對CI與CE進行優化配置。透過適當調整CI間隔,可有效控制節點可傳輸封包的節奏與頻寬分配,進一步降低高負載節點的壓力,提升整體封包處理效率。結合連線參數調整機制,可以有效改善封包壅塞問題,以提升BLE Mesh網路在多節點、高流量情境下的服務品質。

\clearpage

\section{BLE Mesh拓樸建立機制設計}

本論文提出改善FruityMesh的拓樸建立機制,確保Sink節點始終位於整體網路的根節點,並在斷線後能夠自動恢復其角色。此設計為後續的DOT排程機制提供穩定的基礎,如圖\ref{fig: 提出的拓樸建立流程圖}所示。

此外,\cite{108TIT00392035}的研究中對 FruityMesh 原生拓樸建立流程進行了詳盡分析,其內容可作為本論文所提出改進策略的對照與參考資料。

圖\ref{fig: 提出的拓樸建立流程圖}中也針對有底色的部分進行修改,可以更加確保Sink節點始終位於整體網路的根節點,如Algorithm \ref{alg: CalculateClusterScoreAsMaster} 及 Algorithm \ref{alg: CalculateClusterScoreAsSlave}所示。

\begin{figure}[H]
    \centering
    \includegraphics[width = 1\textwidth]{image/build-up_pro2.png}
    \caption{提出的拓樸建立流程圖}
    \label{fig: 提出的拓樸建立流程圖}
\end{figure}

在拓樸建立流程啟動時,每個節點在開機後會廣播包含自身節點資訊的JOIN\_ME封包。節點收到其他節點所廣播的JOIN\_ME封包後,會將該封包儲存於本地緩衝區中,並根據封包內容進行評分(scoring),以評估對方作為連線對象的適合程度。此評分機制可根據鄰近程度、RSSI訊號強度、是否為Sink節點等參數進行加權計算,從而判斷潛在連線對象的優先順序。

若節點本身為預先指定的Sink節點,則會主動嘗試以Master的角色發起連線,以確保其成為拓樸的根節點。另一方面,對於非Sink節點,為了確保拓樸架構的根節點為Sink節點,節點在評估過程中會優先以Slave身分尋找合適的Master進行連線。只有當未能找到合適的Master節點(例如評分皆為0或未收到有效的JOIN\_ME封包)時,該節點才會轉而作為Master嘗試連線至其他尚未建立連線的節點,進一步擴展網路的拓樸結構。

若節點在轉為Master之後,仍未能找到合適的Slave節點(即緩衝區中所有JOIN\_ME封包的評分皆為0或無法建立穩定連線)時,該節點將回到初始狀態,繼續定期廣播JOIN\_ME 封包,以等待後續更合適的連線機會出現。此重試機制能避免節點因短期內找不到連線對象而陷入孤立狀態,提升整體拓樸建立的成功率與網路的自組織能力。

\subsection{決定最好的群組加入}

Algorithm \ref{alg:determine_best_cluster_part1}及Algorithm \ref{alg:determine_best_cluster_part2}及Algorithm \ref{alg:determine_best_cluster_part3}描述了節點在藍牙低功耗網狀網路(BLE Mesh)中,決定其與其他節點建立連線方式的邏輯流程。其核心目標是依據當下網路狀態與節點角色,判斷該節點應該主動發起連線(作為Master)或是被動接受連線(作為Slave)。

\begin{algorithm}[H]
\caption{DetermineBestClusterAvailable - Part 1}
\label{alg:determine_best_cluster_part1}
\begin{algorithmic}[1]
\State Initialize \texttt{result} as \texttt{NO\_NODES\_FOUND}
\State Get device type and assign to \texttt{deviceType}
\If{deviceType is \texttt{SINK} and outbound connections are available}
    \State \texttt{bestClusterAsMaster} $\gets$ \texttt{DetermineBestClusterAsMaster()}
    \If{\texttt{bestClusterAsMaster} $\neq$ null}
        \State Adjust \texttt{connectionIv} based on peer’s device type
        \State Attempt to connect as Master
        \If{connection succeeds}
            \State Update connection attempt time and count
        \EndIf
        \State Set \texttt{result.result} $\gets$ \texttt{CONNECT\_AS\_MASTER}
        \State Set \texttt{result.preferredPartner} $\gets$ bestClusterAsMaster.sender
        \State \Return result
    \EndIf
\EndIf
\end{algorithmic}
\end{algorithm}

Algorithm \ref{alg:determine_best_cluster_part1}流程一開始會檢查當前節點是否為SINK且具備可用的對外連線資源。若節點為Sink節點,則優先考慮作為Master嘗試連線,並透過DetermineBestClusterAsMaster()函式選出最合適的連線目標節點。若成功取得候選節點,系統會根據對方裝置的類型調整連線參數(如CI),並嘗試建立連線。若連線成功,系統會更新連線時間與次數統計,並回傳連線成功的決策結果。

\begin{algorithm}[H]
\caption{DetermineBestClusterAvailable - Part 2}
\label{alg:determine_best_cluster_part2}
\begin{algorithmic}[1]
    \State Reset \texttt{currentAckId} to 0
    \State \texttt{bestClusterAsSlave} $\gets$ \texttt{DetermineBestClusterAsSlave()}

    \If{deviceType is \texttt{SINK}}
        \State \texttt{bestClusterAsSlave} $\gets$ null
    \EndIf

    \If{\texttt{bestClusterAsSlave} $\neq$ null}
        \State \texttt{currentAckId} $\gets$ bestClusterAsSlave.clusterId
        \If{\texttt{meshMaxInConnections == 1}}
            \State Check if any fresh connection exists (handshake not expired)
            \If{no fresh connection and \texttt{freeMeshInConnections == 0}}
                \If{clusterSize $\neq$ bestClusterAsSlave.clusterSize OR random trigger passed}
                    \State Force disconnect other mesh connections
                    \State Reset cluster size to 1
                    \State Generate new \texttt{clusterId}
                \EndIf
            \EndIf
        \EndIf
        \State Update JoinMe packet
        \State Set \texttt{result.result} $\gets$ \texttt{CONNECT\_AS\_SLAVE}
        \State Set \texttt{result.preferredPartner} $\gets$ bestClusterAsSlave.sender
        \State \Return result
    \EndIf
\end{algorithmic}
\end{algorithm}

Algorithm \ref{alg:determine_best_cluster_part2}為若節點角色並不是Sink節點,則會優先作為Slave接受他人連線。此時會重設回應欄位(currentAckId),並透過DetermineBestClusterAsSlave()選出合適的候選節點。並再次檢查當前節點如果為SINK節點,則不允許Sink節點作為Slave進行連線。當節點作為Slave進行連線時,若符合條件的Master節點存在,系統則會建立連線。

\begin{algorithm}[H]
\caption{DetermineBestClusterAvailable - Part 3}
\label{alg:determine_best_cluster_part3}
\begin{algorithmic}[1]
\State \texttt{bestClusterAsMaster} $\gets$ \texttt{DetermineBestClusterAsMaster()}
\If{\texttt{bestClusterAsMaster} $\neq$ null and outbound connections are available}
    \State Adjust \texttt{connectionIv} and attempt to connect
    \If{connection succeeds}
        \State Update connection info
    \EndIf
    \State Set \texttt{result.result} $\gets$ \texttt{CONNECT\_AS\_MASTER}
    \State Set \texttt{result.preferredPartner} $\gets$ bestClusterAsMaster.sender
    \State \Return result
\EndIf

\State Log `no cluster found'
\State Set \texttt{result.result} $\gets$ \texttt{NO\_NODES\_FOUND}
\State \Return result
\end{algorithmic}
\end{algorithm}

Algorithm \ref{alg:determine_best_cluster_part3}為當節點作為Slave進行連線沒有找到適合的Master時,則該節點會再度嘗試以Master身分建立連線,重複執行候選節點評估與連線流程,若成功,則回傳連線結果。最後如果所有嘗試皆沒有辦法成功連線,系統會記錄「未找到叢集」的訊息,並回傳 NO\_NODES\_FOUND 的決策結果。

\subsection{以Mater身分選擇最佳的Slave}

\begin{algorithm}[H]
\caption{CalculateClusterScoreAsMaster}
\label{alg: CalculateClusterScoreAsMaster}
\begin{algorithmic}[1]
\State Retrieve device type from device configuration
\If{packet is too old} \Return 0 \EndIf
\If{packet.clusterId == this.clusterId} \Return 0 \EndIf
\If{packet has no free inbound connections} \Return 0 \EndIf
\If{packet.ackField $\neq$ this.clusterId and $\neq$ 0} \Return 0 \EndIf
\If{packet.clusterSize $>$ current cluster size \textbf{and} current device is not SINK} \Return 0 \EndIf
\If{packet is temporarily blacklisted} \Return 0 \EndIf
\If{already connected to packet.sender} \Return 0 \EndIf
\If{packet.rssi $<$ threshold} \Return 0 \EndIf
\If{current device type == LEAF} \Return 0 \EndIf

\State $rssiScore \gets 100 + packet.rssi$
\State $score \gets packet.freeMeshOutConnections + rssiScore$
\State Modify score using preferred partner policy
\State \Return score
\end{algorithmic}
\end{algorithm}

Algorithm\ref{alg: CalculateClusterScoreAsMaster}說明了節點在嘗試成為Master時,如何針對鄰近的叢集候選節點進行評分。該評分機制用來決定是否值得主動與某個節點建立連線。

一開始,節點會先取得自身的裝置類型,若為LEAF(葉節點)則無法發起連線,直接回傳0分。
接著系統會篩選掉以下不合適的候選節點:

\begin{itemize}
    \item 候選節點的封包太舊
    \item 與候選節點的叢集相同
    \item 候選節點無inbound connections的數量可以使用
    \item 與候選節點的ack欄位不符
    \item 候選節點不是Sink節點且對方叢集較大
    \item 候選節點在短時間內已多次連線失敗(暫時黑名單)
    \item 已與該節點連線中
    \item 候選節點的RSSI過低
\end{itemize}

若符合條件,則將計算RSSI分數,並依據以下項目計算總分:
\begin{itemize}
    \item 候選節點的freeMeshOutConnections數量。
    \item 實際RSSI品質分數(rssiScore)。
\end{itemize}

該分數越高,代表候選節點作為目前節點選擇最佳 Slave 的依據,最後,還會根據偏好節點(preferred partner)進行微調後,回傳最終評分結果。

\subsection{以Slave身分選擇最佳的Master}

\begin{algorithm}[H]
\caption{CalculateClusterScoreAsSlave}
\label{alg: CalculateClusterScoreAsSlave}
\begin{algorithmic}[1]
\State Retrieve device type from configuration
\If{deviceType == SINK} \Return 0 \EndIf
\If{packet.hopsToSink < 0} \Return 0 \EndIf
\If{packet.freeMeshOutConnections == 0} \Return 0 \EndIf
\If{packet is too old} \Return 0 \EndIf
\If{packet.clusterId == current.clusterId} \Return 0 \EndIf
\If{packet.clusterSize < current cluster size \textbf{and} packet.deviceType $\neq$ SINK} \Return 0 \EndIf
\If{packet.rssi < threshold} \Return 0 \EndIf

\State $rssiScore \gets 100 + packet.rssi$
\State $score \gets 0$
\If{packet.deviceType == SINK}
    \State $score \gets score + 10000$
\EndIf

\State $score \gets score + (packet.clusterSize \times 100) + (packet.freeMeshOutConnections \times 100) + rssiScore$
\State Modify score based on preferred partner logic
\State \Return score
\end{algorithmic}
\end{algorithm}

Algorithm\ref{alg: CalculateClusterScoreAsSlave}描述了節點在考慮作為 Slave 加入其他叢集時,如何針對潛在的 Master 節點進行評分。

首先系統會排除不適合的情況:
\begin{itemize}
    \item 若目前節點為SINK節點,則無法作為Slave加入其他叢集。
    \item 候選節點的hopsToSink小於0。
    \item 候選節點無freeMeshOutConnections的數量可以使用。
    \item 候選節點的封包太舊。
    \item 當前節點與候選節點為同一叢集。
    \item 候選節點並非SINK節點且候選節點叢集規模小於當前節點叢集。
    \item 候選節點的RSSI過低。
\end{itemize}

若符合條件,則將計算RSSI分數,並依據以下項目計算總分:

\begin{itemize}
    \item 候選節點是否為SINK節點,若是Sink節點會獲得額外的10000分。
    \item 對方叢集大小(clusterSize)乘以100。
    \item 可用的 freeMeshOutConnections 數量乘以100。
    \item 實際 RSSI 品質分數(rssiScore)
\end{itemize}

最終分數會再次根據偏好節點機制進行修正,分數最高代表候選節點作為目前節點最佳 Master 的依據。

\subsection{FruityMesh 節點間的握手機制}
在 FruityMesh 網狀網路中,節點之間的連線建立採用四階段握手機制。該握手流程由主動發起連線的節點呼叫 StartHandshakeAfterMtuExchange() 函式開始,以下以Node A 向 Node B 發起連線為例,完整流程如下:
\begin{itemize}
    \item Node A 發起 CLUSTER\_WELCOME:Node A 於完成 MTU negotiation 後發送 CLUSTER\_WELCOME 封包至目標節點 Node B,其中包含其叢集資訊(如 clusterId、clusterSize、hopsToSink 等)。
    \item Node B 回覆 CLUSTER\_ACK\_1:Node B 收到封包後進行一系列叢集條件驗證(如 cluster ID 是否相同、cluster size 比較、networkId 是否相符等),若判定可接受連線,則回覆 CLUSTER\_ACK\_1 封包,並設定回傳的 hopsToSink。
    \item Node A 回覆 CLUSTER\_ACK\_2:Node A 接收到 ACK\_1 封包後,確認對方已接受連線並願意合併叢集,因此發送 CLUSTER\_ACK\_2 封包回應,作為握手完成的最終步驟。
    \item Node B 接收 ACK\_2,Handshake 完成:Node B 收到 ACK\_2 後,將當前連線視為正式有效,至此雙方完成 Mesh 連線建構並合併拓樸結構。
\end{itemize}

\subsection{Self-Healing機制的拓樸修復改進}

原生FruityMesh提供了Self-Healing機制,使節點在斷線後能回到HIGH\_DISCOVERY狀態並尋找可連線的鄰居節點,且會由小的叢集合併進大的叢集,從而重新建立連線。然而,此機制下若斷線的是Root(Sink)節點,重新連回後將失去Root身份,可能導致整個網路拓樸失效或性能下降。

為解決此問題,本研究修改了StartHandshakeAfterMtuExchange 以及 CLUSTER\_WELCOME 訊息的握手流程。在握手判斷邏輯中,若發現對方為SINK且其Cluster比自己小(或等於),則避免讓原本為Root的節點退位為普通節點,並強制原Cluster切斷其他連線,重新啟動拓樸建立,使原本的SINK節點能保有Root身份,維持拓樸穩定性與網路一致性,如Algorithm\ref{alg: StartHandshakeAfterMTUExchange} 以及 Algorithm\ref{alg: ClusterWelcomePacketReceived} 所示。

\begin{algorithm}[H]
\caption{Start Handshake After MTU Exchange}
\label{alg: StartHandshakeAfterMTUExchange}
\begin{algorithmic}[1]
\Require Connection ready after MTU exchange
\State Backup local \texttt{clusterID} and \texttt{clusterSize}
\State Clear current cluster info update packet
\If{connectionState $\geq$ HANDSHAKING}
    \State Log and \Return \Comment{Handshake already in progress or finished}
\EndIf
\State Log ``Handshake starting''
\State Retrieve \texttt{deviceType} from hardware configuration
\State Set \texttt{connectionState} = HANDSHAKING
\State Record handshake start timestamp
\State Build \texttt{CLUSTER\_WELCOME} packet with:
    \begin{itemize}
        \item \texttt{clusterId = backupClusterID}
        \item \texttt{clusterSize = backupClusterSize}
        \item \texttt{deviceType}
        \item \texttt{hopsToSink = GetHopsToNearestSink()}
        \item \texttt{preferredConnectionInterval = 0}
        \item \texttt{networkId = local networkId}
    \end{itemize}
\If{\texttt I am not SINK cluster}
    \State disconnect all other mesh connections
    \State Reinitialize local cluster with clusterSize = 1 and newCluster ID
    \State Do not send packet
\Else
    \State Log packet info (clusterId, clusterSize, hopsToSink)
    \State Send handshake message with \texttt{CLUSTER\_WELCOME} packet
\EndIf
\end{algorithmic}
\end{algorithm}

在拓樸自我修復(Topology Self-Healing)機制中,節點於 MTU (Maximum Transmission Unit) 完成後會立即觸發握手流程(Handshake)。其主要流程如演算法\ref{alg: StartHandshakeAfterMTUExchange} 所示,首先節點會備份當前的叢集識別碼(clusterID)與叢集大小(clusterSize),並清除現有的叢集資訊更新封包,以確保後續判斷的數據一致性。若連線狀態(connectionState)已經處於或高於 HANDSHAKING 狀態,表示握手程序正在進行或已完成,此時將記錄相關日誌並直接結束流程,以避免重複執行。

接著,節點會透過硬體介面讀取當前裝置類型(deviceType),並將連線狀態設定為 HANDSHAKING,同時記錄握手開始的時間戳記。之後,節點會依據備份的叢集資訊建立 CLUSTER\_WELCOME 封包,其中包含叢集識別碼、叢集大小、裝置類型、與最近 Sink 節點的跳數(hopsToSink)以及網路識別碼(networkId)。

當 本節點的Cluster 是不包含 Sink 節點的叢集時,代表目前叢集並沒有Sink節點。此時節點會主動斷開其他所有 Mesh 連線,並重新初始化自身的叢集資訊(clusterSize = 1 並產生新的 clusterID),且不會發送任何握手封包。

若 本節點的Cluster 是包含 Sink 節點的叢集時,則節點會先記錄封包資訊(包含 clusterId、clusterSize 與 hopsToSink),並發送 CLUSTER\_WELCOME 封包給目標節點,以完成握手啟動程序。


\begin{algorithm}[H]
\caption{CLUSTER\_WELCOME packet received}
\label{alg: ClusterWelcomePacketReceived}
\begin{algorithmic}[1]
\Require Received \texttt{CLUSTER\_WELCOME} packet
\If{packet size invalid}
    \State Log and ignore packet
\Else
    \State Save partner handle and enter HANDSHAKING state
    \State Backup local cluster ID and size
    \If{remote cluster ID == local cluster ID}
        \State Disconnect: SAME\_CLUSTERID
    \ElsIf{remote cluster size < local size \textbf{and} remote is not SINK}
        \If{connection is inbound}
            \State Disconnect: WRONG\_DIRECTION
        \EndIf
    \ElsIf{network ID mismatch}
        \State Disconnect: NETWORK\_ID\_MISMATCH
    \ElsIf{connection is not preferred \textbf{and} preferences ignored}
        \State Disconnect: UNPREFERRED\_CONNECTION
    \Else
        \State Save partner nodeId
        \State Build and send \texttt{CLUSTER\_ACK\_1}:

        \State Disconnect all other mesh connections
        \State Reinitialize local cluster with clusterSize = 1 and newCluster ID
    \EndIf
\EndIf
\end{algorithmic}
\end{algorithm}

Algorithm\ref{alg: ClusterWelcomePacketReceived}為了強化FruityMesh在節點斷線重連後的拓樸穩定性,本研究在原有 Self-Healing機制之上,進一步修改其握手邏輯。當節點收到 CLUSTER\_WELCOME 封包後,將進入拓樸重組的判斷流程。具體邏輯如下:

\begin{itemize}
    \item 首先,節點會確認封包的大小是否正確,若不正確則忽略該封包。
    \item 接著,進入Handshaking狀態並備份目前的Cluster ID與Cluster大小,用以後續比對使用。
    \item 若對方節點的Cluster ID與自己相同,代表已屬同一個Cluster,這在初始階段是不應該出現的情況,因此直接斷線處理。
    \item 若對方的 Cluster大小比本地端小,且其並非SINK節點,則本節點認定自己應作為主導方,若該連線是inbound,則代表方向錯誤,同樣會中止連線。
    \item 若兩節點的network ID不一致,或連線對象並非偏好的節點,也將被強制中止以避免不必要的Mesh錯誤連線。
\end{itemize}

在以上檢查皆通過的情況下,代表本節點接受加入對方節點的 Cluster,並切斷原本Cluster的連線。並發送 CLUSTER\_ACK\_1 封包進行回覆,繼續後續的握手流程。

\section{BLE Mesh封包傳輸機制設計}
在 BLE Mesh 網路中,封包傳輸的效率與穩定性對整體系統效能具有關鍵影響。尤其在具備多跳(multi-hop)特性的網狀架構中,位於拓樸上層的節點(例如接近 Root 的節點)需承擔來自下層節點的大量封包轉送任務,因此其封包處理壓力遠高於葉節點。若未妥善配置這些節點的傳輸參數,將容易造成 sent buffer 過載、傳輸延遲增加,甚至出現頻繁重傳等問題,進而影響整體系統穩定性與效能表現。

為了深入分析封包轉送壓力與重傳率之間的關係,本研究首先在 Mesh 網路中以固定的傳輸間隔進行實驗。在預設狀況下,各節點封包發送間隔為 100 ms,於此設定下觀察到中繼節點經常發生 buffer 壅塞與重傳現象。為進一步驗證其影響,將封包發送間隔調整為 200 ms,結果顯示整體重傳率大幅下降,甚至趨近於 0,顯示延長封包發送間隔可有效降低節點傳輸壓力。

然而,過長的封包發送間隔可能導致資料無法即時傳輸,影響應用效能。因此,本研究並未僅以延長發送間隔作為最終解法,而是進一步針對現有 100 ms 間隔條件下,設計一種「層級式連線間隔調整機制」,以在維持即時性前提下,緩解中繼節點的壓力並改善整體傳輸表現。

為了改善上述問題,本研究提出一種基於節點層級調整連線間隔參數(Connection Interval, CI)的機制,如圖 \ref{fig: BLE Mesh封包傳輸機制設計節點架構圖} 所示。設計理念為:根據節點於拓樸中的層級深度,賦予不同的 CI 值,越靠近 Root 的節點,其連線間隔設定越短,以提供更頻繁的傳輸機會。此方法能夠有效緩解中繼節點所面臨的頻寬壓力,減少傳輸阻塞與重傳情形,進一步提升整體 BLE Mesh 網路的穩定性與資料傳遞效率。

\begin{figure}[H]
    \centering
    \includegraphics[width = 0.3\textwidth]{image/BLE Mesh封包傳輸機制設計節點架構圖.png}
    \caption{BLE Mesh封包傳輸機制設計節點架構圖}
    \label{fig: BLE Mesh封包傳輸機制設計節點架構圖}
\end{figure}

圖 \ref{fig: BLE Mesh封包傳輸機制設計節點架構圖} 顯示一個包含六個節點的 Mesh 拓樸,其中節點 1 為 Root。依據拓樸深度,每層的 CI 設定分別為 25 ms、50 ms 與 100 ms,形成由上而下頻率遞減的調整策略,有效反映節點所承擔的資料轉發負載,透過此層級化的參數配置機制,本研究期望在不增加硬體資源的前提下,提升多跳 BLE Mesh 網路在長時間運行中的穩定性與封包抵達率。

\end{ZhChapter}